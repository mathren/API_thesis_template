% !TEX root =  thesis.tex

% The propositions, traditionally on a separate piece of paper. 
% Six of them must be about your research!

% For sending them to ipskamp
\documentclass[12pt]{report}

% Some options for printing them yourself (A5 format, to fit 2 on an A4)
%\documentclass[10pt,a5paper]{report}
%%\usepackage[a5paper,textwidth=120mm,textheight=170mm]{geometry}
%\usepackage[a5paper,textwidth=110mm,textheight=160mm]{geometry}

\usepackage[dutch,english]{babel}

\begin{document}\pagestyle{empty}

% propositions in english
\selectlanguage{english}

\begin{center}
{\Large \bf Propositions belonging to this thesis}
\vspace{10mm}
\end{center}

\begin{enumerate}
\item First proposition. (Chapters 2, 3, 4 \& 5)
\item Another one. (Chapter 4)
\item Another one. (Chapter 3)
\item Another one. (Chapter 6)
\item Another one. (Chapter 2)
\item Another one. (Chapter 2)

\item Here you can start with propositions not related to your research.
\item Or just funny ones.
\item Another one.
\item Another one.
\item Traditionally, the 11th and final proposition is a bogus one, like ``it's going to rain'',``this thesis contains spelling errors'' or ``change is a constant factor''
\end{enumerate}

% propositions in Dutch, on the back
\clearpage
\selectlanguage{dutch}

\begin{center}
{\Large \bf Stellingen behorende bij dit proefschrift}
\vspace{10mm}
\end{center}
\begin{enumerate}
\item Dezelfde stellingen, maar nu in het Nederlands. (Chapters 2, 3, 4 \& 5)
\item Idem. (Chapter 4)
\item Idem. (Chapter 3)
\item Idem. (Chapter 6)
\item Idem. (Chapter 2)
\item Idem. (Chapter 2)
\item Idem.
\item Idem.
\item Idem.
\item Idem.
\item Bogus.
\end{enumerate}

\end{document}
